\documentclass[10pt]{article}

%Math
\usepackage{amsmath}
\usepackage{amsfonts}
\usepackage{amssymb}
\usepackage{amsthm}
\usepackage{ulem}
\usepackage{stmaryrd} %f\UTF{00FC}r Blitz!

%PageStyle
\usepackage[ngerman]{babel} % deutsche Silbentrennung
\usepackage[utf8]{inputenc} 
\usepackage{fancyhdr, graphicx}
\usepackage[scaled=0.92]{helvet}
\usepackage{enumitem}
\usepackage{parskip}
\usepackage[a4paper,top=2cm]{geometry}
\setlength{\textwidth}{17cm}
\setlength{\oddsidemargin}{-0.5cm}


% Shortcommands
\newcommand{\Bold}[1]{\textbf{#1}} %Boldface
\newcommand{\Kursiv}[1]{\textit{#1}} %Italic
\newcommand{\T}[1]{\text{#1}} %Textmode
\newcommand{\Nicht}[1]{\T{\sout{$ #1 $}}} %Streicht Shit durch

%Arrows
\newcommand{\lra}{\leftrightarrow} 
\newcommand{\ra}{\rightarrow}
\newcommand{\la}{\leftarrow}
\newcommand{\lral}{\longleftrightarrow}
\newcommand{\ral}{\longrightarrow}
\newcommand{\lal}{\longleftarrow}
\newcommand{\Lra}{\Leftrightarrow}
\newcommand{\Ra}{\Rightarrow}
\newcommand{\La}{\Leftarrow}
\newcommand{\Lral}{\Longleftrightarrow}
\newcommand{\Ral}{\Longrightarrow}
\newcommand{\Lal}{\Longleftarrow}

% Code listenings
\usepackage{color}
\usepackage{xcolor}
\usepackage{listings}
\usepackage{caption}
\DeclareCaptionFont{white}{\color{white}}
\DeclareCaptionFormat{listing}{\colorbox{gray}{\parbox{\textwidth}{#1#2#3}}}
\captionsetup[lstlisting]{format=listing,labelfont=white,textfont=white}
\lstdefinestyle{JavaStyle}{
 language=Java,
 basicstyle=\footnotesize\ttfamily, % Standardschrift
 numbers=left,               % Ort der Zeilennummern
 numberstyle=\tiny,          % Stil der Zeilennummern
 stepnumber=5,              % Abstand zwischen den Zeilennummern
 numbersep=5pt,              % Abstand der Nummern zum Text
 tabsize=2,                  % Groesse von Tabs
 extendedchars=true,         %
 breaklines=true,            % Zeilen werden Umgebrochen
 frame=b,         
 %commentstyle=\itshape\color{LightLime}, Was isch das? O_o
 %keywordstyle=\bfseries\color{DarkPurple}, und das O_o
 basicstyle=\footnotesize\ttfamily,
 stringstyle=\color[RGB]{42,0,255}\ttfamily, % Farbe der String
 keywordstyle=\color[RGB]{127,0,85}\ttfamily, % Farbe der Keywords
 commentstyle=\color[RGB]{63,127,95}\ttfamily, % Farbe des Kommentars
 showspaces=false,           % Leerzeichen anzeigen ?
 showtabs=false,             % Tabs anzeigen ?
 xleftmargin=17pt,
 framexleftmargin=17pt,
 framexrightmargin=5pt,
 framexbottommargin=4pt,
 showstringspaces=false      % Leerzeichen in Strings anzeigen ?        
}

%Config
\renewcommand{\headrulewidth}{0pt}
\setlength{\headheight}{15.2pt}

%Metadata

\title{
	\vspace{5cm}
	UTF-8 Vorlage
}
\author{Jan Fässler, Jonas Schwammberger}
\date{6. Semester (FS 2014)}


% hier beginnt das Dokument
\begin{document}

\setcounter{page}{1}
\pagenumbering{arabic}

% Inhalt Start

\section{Aufgabe 3}
\begin{tabular}{r|l}
Folgende Zeitmessungen haben wir mit unserem Algorithmus gemacht:
Query Size & Time needed \\\hline \hline
10'000 & 45'489ms\\\hline
20'000 & 77'706ms\\\hline
40'000 & 149'215ms\\\hline
80'000 & 347'096ms\\

\end{tabular}
\section{Aufgabe 3}
Jede Iteration des Algorithmus besteht aus drei Schritten:
\begin{enumerate}
	\item neues Wort in die Wortmengen hinzufügen.
	\item Zählen, wie oft eine Wortmenge vorkommt.
	\item Wortmengen löschen, welche unter dem Support-Value liegen.
\end{enumerate}

Das Löschen kann in linearer Zeit durchgeführt werden, es fällt in der Big-O Notation nicht ins Gewicht.\\
Das Zählen und das Hinzufügen geschehen beides in einer verschachtelten for-Schleife, was eine Laufzeit von $O(n^2)$ vermuten lässt, bei einer genaueren Analyse zeigt sich jedoch ein anderes Bild.
Für die Analyse sei:
\begin{enumerate}
	\item n die Anzahl von Queries
	\item m die Anzahl Wortmengen, die gezählt werden sollen.
	\item p die Anzahl Wörter, welche den Support-Value erfüllen.
	\item q die durchschnittliche Anzahl Wörter in einer Query
\end{enumerate}
Daraus ergibt sich die Laufzeit für das Hinzufügen $m*p$ und für das Zählen $n*m*ld q$. $m$ und $p$ verhalten sich quasi asymptotisch zu den Anzahl Queries. Im Unendlichen betrachtet ist die Laufzeit $O(n)$, da $m$ und $p$. Dieses Verhalten spiegelt sich auch in den Zeitmessungen wieder. Bei einer Verdoppelung der Anzahl Queries wird die Laufzeit beinahe verdoppelt. 


% Inhalt Ende 
\end{document} 